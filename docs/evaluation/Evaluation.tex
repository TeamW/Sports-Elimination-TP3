\documentclass[a4paper,10pt,BCOR10mm,oneside,headsepline]{scrartcl}
\usepackage[ngerman]{babel}
\usepackage[utf8]{inputenc}
\usepackage{wasysym}% provides \ocircle and \Box
\usepackage{enumitem}% easy control of topsep and leftmargin for lists
\usepackage{color}% used for background color
\usepackage{forloop}% used for \Qrating and \Qlines
\usepackage{ifthen}% used for \Qitem and \QItem
\usepackage{typearea}
\areaset{17cm}{26cm}
\setlength{\topmargin}{-1cm}
\usepackage{scrpage2}
\pagestyle{scrheadings}
\ohead{\pagemark}
\chead{}
\cfoot{}

%%%%%%%%%%%%%%%%%%%%%%%%%%%%%%%%%%%%%%%%%%%%%%%%%%%%%%%%%%%%
%% Beginning of questionnaire command definitions         %%
%%%%%%%%%%%%%%%%%%%%%%%%%%%%%%%%%%%%%%%%%%%%%%%%%%%%%%%%%%%%
%%
%% 2010, 2012 by Sven Hartenstein
%% mail@svenhartenstein.de
%% http://www.svenhartenstein.de
%%
%% Please be warned that this is NOT a full-featured framework for
%% creating (all sorts of) questionnaires. Rather, it is a small
%% collection of LaTeX commands that I found useful when creating a
%% questionnaire. Feel free to copy and adjust any parts you like.
%% Most probably, you will want to change the commands, so that they
%% fit your taste.
%%
%% Also note that I am not a LaTeX expert! Things can very likely be
%% done much more elegant than I was able to. If you have suggestions
%% about what can be improved please send me an email. I intend to add
%% good tipps to my website and to name contributers of course.
%%
%% 10/2012: Thanks to karathan for the suggestion to put \noindent
%% before \rule!

%% \Qq = Questionaire question. Oh, this is just too simple. It helps
%% making it easy to globally change the appearance of questions.
\newcommand{\Qq}[1]{\textbf{#1}}

%% \QO = Circle or box to be ticked. Used both by direct call and by
%% \Qrating and \Qlist.
\newcommand{\QO}{$\Box$}% or: $\ocircle$

%% \Qrating = Automatically create a rating scale with NUM steps, like
%% this: 0--0--0--0--0.
\newcounter{qr}
\newcommand{\Qrating}[1]{\QO\forloop{qr}{1}{\value{qr} < #1}{---\QO}}

%% \Qline = Again, this is very simple. It helps setting the line
%% thickness globally. Used both by direct call and by \Qlines.
\newcommand{\Qline}[1]{\noindent\rule{#1}{0.6pt}}

%% \Qlines = Insert NUM lines with width=\linewith. You can change the
%% \vskip value to adjust the spacing.
\newcounter{ql}
\newcommand{\Qlines}[1]{\forloop{ql}{0}{\value{ql}<#1}{\vskip0em\Qline{\linewidth}}}

%% \Qlist = This is an environment very similar to itemize but with
%% \QO in front of each list item. Useful for classical multiple
%% choice. Change leftmargin and topsep accourding to your taste.
\newenvironment{Qlist}{%
  \renewcommand{\labelitemi}{\QO}
  \begin{itemize}[leftmargin=1.5em,topsep=-.5em]
}{%
  \end{itemize}
}

%% \Qtab = A "tabulator simulation". The first argument is the
%% distance from the left margin. The second argument is content which
%% is indented within the current row.
\newlength{\qt}
\newcommand{\Qtab}[2]{
  \setlength{\qt}{\linewidth}
  \addtolength{\qt}{-#1}
  \hfill\parbox[t]{\qt}{\raggedright #2}
}

%% \Qitem = Item with automatic numbering. The first optional argument
%% can be used to create sub-items like 2a, 2b, 2c, ... The item
%% number is increased if the first argument is omitted or equals 'a'.
%% You will have to adjust this if you prefer a different numbering
%% scheme. Adjust topsep and leftmargin as needed.
\newcounter{itemnummer}
\newcommand{\Qitem}[2][]{% #1 optional, #2 notwendig
  \ifthenelse{\equal{#1}{}}{\stepcounter{itemnummer}}{}
  \ifthenelse{\equal{#1}{a}}{\stepcounter{itemnummer}}{}
  \begin{enumerate}[topsep=2pt,leftmargin=2.8em]
  \item[\textbf{\arabic{itemnummer}#1.}] #2
  \end{enumerate}
}

%% \QItem = Like \Qitem but with alternating background color. This
%% might be error prone as I hard-coded some lengths (-5.25pt and
%% -3pt)! I do not yet understand why I need them.
\definecolor{bgodd}{rgb}{0.8,0.8,0.8}
\definecolor{bgeven}{rgb}{0.9,0.9,0.9}
\newcounter{itemoddeven}
\newlength{\gb}
\newcommand{\QItem}[2][]{% #1 optional, #2 notwendig
  \setlength{\gb}{\linewidth}
  \addtolength{\gb}{-5.25pt}
  \ifthenelse{\equal{\value{itemoddeven}}{0}}{%
    \noindent\colorbox{bgeven}{\hskip-3pt\begin{minipage}{\gb}\Qitem[#1]{#2}\end{minipage}}%
    \stepcounter{itemoddeven}%
  }{%
    \noindent\colorbox{bgodd}{\hskip-3pt\begin{minipage}{\gb}\Qitem[#1]{#2}\end{minipage}}%
    \setcounter{itemoddeven}{0}%
  }
}

%%%%%%%%%%%%%%%%%%%%%%%%%%%%%%%%%%%%%%%%%%%%%%%%%%%%%%%%%%%%
%% End of questionnaire command definitions               %%
%%%%%%%%%%%%%%%%%%%%%%%%%%%%%%%%%%%%%%%%%%%%%%%%%%%%%%%%%%%%

\begin{document}

\begin{center}
  \textbf{\huge Team W User Evaluation}
\end{center}\vskip1em

\noindent Thank you for participating in our Team Project User
Evaluation. Please fill in the below questionnaire when prompted to do
so by the team member doing the evaluation. \\

You are asked to use this prototype in a ``think-aloud'' manner so we
can futher understand a users intuition when using this
prototype. Please note that you are not being evaluated and that we
are only testing the usability and accuracy of our system. \\

You have the right to stop this evaluation, with no requirement to give
reason, at any time. If you have any further thoughts or questions
about any part of the user evaluation, please contact James
Gallagher at  0800899g@student.gla.ac.uk .

\section*{Desktop Evaluation}
\vskip.5em

\Qitem{\Qq{Find the team who finished top of the \emph{East Division}
    of the \emph{American League} on the last day of the season
    (October 4 2012).}}
{\Qtab{4cm}{Team: \Qline{10cm}}}

\Qitem{\Qq{Find all the teams in the \emph{National Central League}
    who are 102 games into the season by August 1 2012.}}
{\Qtab{4cm}{Team(s): \Qline{10cm}}}

\Qitem{\Qq{What are the names of the teams that have been eliminated
    in the \emph{American West League} five days before the end of
    the season?}}
{\Qtab{4cm}{Team(s): \Qline{10cm}}}

\Qitem{\Qq{What are the names of the teams that have been eliminated
    in the \emph{National West League} on July 12 2012.}}
{\Qtab{4cm}{Team(s): \Qline{10cm}}}

\Qitem{\Qq{Print this screen using the applications print to PDF
    functionality.}}

\Qitem{\Qq{Add this screen and the next three days to the print queue,
    and then print this document.}} 

\Qitem{\Qq{Generate a new league, and load this league into the
    system.}} 

\section*{Web Evaluation}

\minisec{Open your desired web browser and navigate to ``www.gordonrenfrewshire.com/teamw''}

\vskip.5em

\Qitem{\Qq{Find the team who finished top of the \emph{East Division}
    of the \emph{American League} on the last day of the season
    (October 4 2012).}}
{\Qtab{4cm}{Team: \Qline{10cm}}}

\Qitem{\Qq{Find all the teams in the \emph{National Central League}
    who are 102 games into the season by August 1 2012.}}
{\Qtab{4cm}{Team(s): \Qline{10cm}}}

\Qitem{\Qq{What are the names of the teams that have been eliminated
    in the \emph{American West League} five days before the end of
    the season?}}
{\Qtab{4cm}{Team(s): \Qline{10cm}}}

\Qitem{\Qq{What are the names of the teams that have been eliminated
    in the \emph{National West League} on July 12 2012.}}
{\Qtab{4cm}{Team(s): \Qline{10cm}}}

\end{document}
