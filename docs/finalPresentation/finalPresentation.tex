\documentclass{beamer}
\newcommand\measurepage{\dimexpr\pagegoal-\pagetotal-\baselineskip\relax}
\title{Team W - Algorithms for Sports Eliminations}
\author{
    Gordon Reid: 1002536R\\
    Ryan Wells: 1002253W\\
    Kris Stewart: 1007175S\\
    David Selkirk: 1003646S\\
    James Gallagher: 0800899G\\
    Dr David Manlove: Project Supervisor
}
\date{March 19, 2013}
\begin{document}
\frame{\titlepage}
%"Presentations should describe the aims and objectives of the project, the
%work you have undertaken and the resulting product, the key design and
%management decisions you made, and any lessons learnt. Tell us what you’ve
%done, why it was interesting, what you learnt, what you’d do differently if you
%had to do it again, etc."

% Introduction - Ryan
% Parser - Kris
% Desktop UI - David
% Web UI - Gordon
% Algorithm - Ryan
% User Eval - James
% Conclusion - Gordon
\frame{
  \frametitle{Project introduction}
  \begin{itemize}
  \item Project aims to answer sports elimination question.
  \item What is sports elimination?
  \item Main modules: Parsing, Desktop/Web User Interface, Algorithm, User Evaluation
  \item Builds on na\"{\i}ve calculations made by sports pundits.
  \end{itemize}
}
\frame{
  \frametitle{Sport}
  \begin{itemize}
  \item American Major League Baseball.
  \item 1 point for a win, 0 points for a loss, no draws allowed.
  \item Six divisions with four to six teams per division.
  \end{itemize}
}
\frame{
  \frametitle{Example}
  \begin{center}
  \begin{figure}
  \includegraphics[width=\textwidth]{Example.png}
  \end{figure}
  \tiny{Image courtesy of Kevin D. Wayne, Princeton University}
  \end{center}
  \begin{itemize}
  \item Montreal trivially eliminated
  \item Philadelphia non-trivially eliminated
  \end{itemize}
}
\frame{
  \frametitle{Parser}
  \pause
  \begin{center}
  \begin{figure}
  \includegraphics[width=\textwidth]{textfile.png}
  \end{figure}
  \end{center}
  \begin{itemize}
  \item<2-> Uses basic string comparison to put the relevant team in the
  correct League and Division.
  \item<3-> Have considered future plans to parse a web page.
  \item<4-> Our current source is not suitable for real-time updating.
  \item<5-> As a team we are unsure if we would like to implement this feature
  or concentrate our energies on other features for example, making a web-based
  UI.
  \end{itemize}
}
\frame{
  \frametitle{User Interface}
  \begin{itemize}
  \item<2-> User is likely an advanced follower of the sport looking for
  very specific information.
  \item<2-> Simple design based on presenting this key information immediately.
  \item<3-> More detailed information available on user request.
  \item<4-> Challenge is to integrate the two tasks such that the UI is
  both functionally and aesthetically pleasing.
  \end{itemize}
}
\frame{
  \frametitle{User Interface - Key Tasks}
  \begin{itemize}
  \item<1-> Display all teams by division and league, showing elimination status.
  \item<2-> Update the elimination status on start-up.
  \item<3-> Display certification of elimination for teams eliminated.
  \end{itemize}
}
\frame{
  \frametitle{User Interface - Prototype}
  \begin{figure}
  \includegraphics[width=\textwidth]{InitialUI.png}
  \end{figure}
}
\frame{
  \frametitle{User Interface - Additional Tasks}
  \begin{itemize}
  \item<1-> Allow date navigation, displaying correct results for each the date.
  \item<2-> Display print functionality.
  \item<3-> Display league generation functionality.
  \end{itemize}
}
\frame{
  \frametitle{User Interface - Final}
  \begin{figure}
  \includegraphics[scale=0.4,keepaspectratio]
  {finalDesktopUI.png}
  \end{figure}
}
\frame {
  \frametitle{Web Application}
  \begin{itemize}
  \item<2-> An extension of the project.
  \item<2-> Contains a subset of most important functionality of
  desktop application.
  \item<2-> Numerous extensions of web application part of future work.
  \end{itemize}
}
\frame {
  \frametitle{Web Application Screenshot}
  \includegraphics[width=\textwidth,height=\measurepage,keepaspectratio]
  {webAppScreenshot.png}
}
\frame{
  \frametitle{Algorithm}
  \pause
  \begin{itemize}
  \item<2-> Revolves around creation of directed graphs and flow
    through directed graphs.
  \item<3-> Pushing flow through a specific graph will allow us to
    determine if the given team X is eliminated from a given league.
  \item<4-> A team X can win the league if it is possible for there to
    be a saturating flow. A flow is saturated if the total flow from
    the source equals the total capacity from the source.
  \item<5-> All flow leaving the source has to arrive at the sink.
  \end{itemize}
}
\frame{
  \frametitle{Algorithm - Graph}
  \begin{center}
  \begin{figure}
  \includegraphics[width=0.6\textwidth]{graph.png}
  \end{figure}
  \tiny{Image courtesy of Kevin D. Wayne, Princeton University}
  \end{center}
  \begin{itemize}
  \item $team_k$ - Team being tested for elimination.
  \item $team_i$ and $team_j$ - Arbitrary teams in $team_k$'s division
  \item $g_{i, j}$ - Number of games left between $team_i$ and $team_j$
  \item W - Current score of $team_k$ + games $team_k$ has remaining
  \item $w_{i}$ - Number of game $team_i$ won
  \end{itemize}
}
\frame{
  \frametitle{Algorithm - Extensions}
  \begin{itemize}
    \item<2-> League Elimination by Binary Search Computation
    \item<3-> Certificate of Elimination
    \item<4-> First non-trivially eliminated team
  \end{itemize}
}
\frame {
\frametitle{ User Evaluation }
\begin{itemize}
\item<2-> Aim : Quality feedback regarding the useability and functionality supported of the system we create.
\item<3->Structure :
Stage 1 : Participant Breif (Intro)
Stage 2 : Task Script (Useability)
Stage 3 : Questionare (Likeability)
\end{itemize}
}
\frame{
\frametitle{Supprizes }
\begin{itemize}
\item<2->More intersting points !
\item<3->Tangible results very quickly.
\item<4-> Design faults tripped up the most experienced of users.
\item<5->Very wide variety of attention span in test group .
\item<6-> User's got tripped up by testing method .
\end{itemize}
}
\frame{
\frametitle{Varied and Quality Feedback !}
 \begin{itemize}
\item<2->Overall, Very Sucessfull
\item<3->Discovered of Bugs, Usability Issues and Design Flaws - Highlight the value of User Testing.
\item<4-> Design Flaw : Radio Button for League and Division selection .
\item<5->Usability Issues : Inadeqet Documentation provided at time of testing/ Missed Functionality !  .
\item<6-> Bug : Date not being displayed on loading of Web application .
\end{itemize}
}
\frame{
\frametitle{Over Very Positive !}
\begin{itemize}
\item<2->System received numerous positive comments on useability and asthetic.
\item<3->Highlighted value of qulaity user testing !
\item<4-> Provided great ideas for future work, design improvements, tweaks . . etc
\item<5-> Lots of users interested in the system, even if had no interest in the domain.
\end{itemize}
}
\frame {
  \frametitle{Lessons Learnt and Future Work}
  Lessons Learnt
  \begin{itemize}
  \item ...
  \end{itemize}
  Future Work
  \begin{itemize}
  \item Complete changes suggested based on user evaluation.
  \item Continue implementation of web application.
  \item Binary first non-trivial elimination search.
  \end{itemize}
}
\frame {
  \frametitle{Project Conclusion}
  \begin{itemize}
  \item Successfully implemented all must have functionality.
  \item Fun and interesting to take theory and implement real-world application.
  \item Informative user evaluation.
  \item Plenty of scope for future work.
  \end{itemize}
}
\frame {
  We would now like to invite questions from the panel.

  Thank you for listening.
}
\end{document}
