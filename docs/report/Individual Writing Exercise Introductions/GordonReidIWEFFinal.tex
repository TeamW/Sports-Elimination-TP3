\documentclass[11pt]{article}

\usepackage[utf8]{inputenc}

\usepackage{hyperref}
\hypersetup{colorlinks=true}

\usepackage{geometry}
\geometry{a4paper}

\title{Team Project - An Introduction\\Algorithms for Sports Eliminations}
\author{Gordon Reid - 1002536r}
\date{\today}

\begin{document}

\maketitle

\section*{Preliminaries}

The project is based around working out if a baseball team has been eliminated
from the division it is currently playing in however no knowledge of baseball
is required. The project can also be extended to some other sports such as 
Basketball. Each division has a small number of teams who play each other 
multiple times. Each match requires one team to win. The winning team earns a 
single point, whilst the losing team does not earn any points. Draws are not
permitted. A team is eliminated if it cannot earn enough points to secure the 
position at the top of the points table. For the project, American leagues
and divisions will be focussed on as baseball is popular in America.

The report makes little use of jargon terms, and has been structured with the
aim of being as accessible as possible. Parts of the report make reference to
terms and concepts where knowledge of Software Engineering, Object Oriented
Programming, Graph Theory, and Network Flow will be useful but not required.

For those readers who may be unfamiliar with Graph Theory, I will briefly
describe the basic structure of graphs, in particular directed graphs.
Graphs are made up of two components, vertices and edges. Vertices can be
thought of as cities, with edges being roads between the cities. A road between
two cities can have a maximum number of cars on it at one time (its capacity)
and a number of cars on it at the present time (its flow). The flow of traffic
can never exceed capacity. The road from city A to city B is separate from the 
road from city B to city A and the existence of one road does not imply the 
existence of the other.

Network Flow is the study of working out how to have as much traffic move
from a start city to an end city by passing through intermediate cities.

There is only one algorithm discussed in the report, this is the Ford-Fulkerson
algorithm. The algorithm involves creating graphs for each team from the 
remaining matches between two other teams in the division and the other teams 
themselves. The team being tested only has a mathematically possible chance of 
winning the league if there is the flow from the special vertex `source' to the 
other special vertex `sink' is saturating. Saturating meaning that the total 
flow out from the `source' is the same as the total capacity out from the 
`source'. If there is no saturating flow then the team has no chance of 
winning. The source vertex is a vertex where there are no incoming edges (start 
city), and the sink vertex is a vertex where there are no outgoing edges (end 
city).

\section*{Outline}

The remainder of the report will cover the background of the project and will
explore the reason why the sports elimination calculation became something of
interest. The background section will also give two references to publications
the project is based and builds on.

After the background material, the report will then discuss the aims of the
project, from the primary goal to the extensions that will be explored if time
permits.

Finally, our motivation for working on this project will be discussed.

\section*{Background}

The project builds on two papers=: one by K. D. Wayne, `A new property and a 
faster algorithm for baseball elimination', SIAM Journal on Discrete 
Mathematics \url{http://epubs.siam.org/doi/abs/10.1137/S0895480198348847} and 
one by I. Adler, A. L. Erera, D.S. Hochbaum and E. V. Olinick, `Baseball,
optimization and the world web', manuscript, 1998
\url{http://riot.ieor.berkeley.edu/~dorit/pub/baseball.ps}.

The project came about from the American love for baseball, and the love for
a multitude of statistics relating to baseball. A common news story for sports
fans is the announcement that a team can no longer win the current season
however the calculations used to work this out are typically na\"{\i}ve and
do not take into account the remaining schedule of games.

The papers stated above, build on the na\"{\i}ve calculations and make use of a
simple, yet effective, algorithm to reliably answer the question:
`Is it mathematically impossible for a given team to win the baseball league?'
The algorithm is known as the Ford-Fulkerson algorithm, named after L. R. Ford,
Jr. and D. R. Fulkerson. The Ford-Fulkerson algorithm removes the need for an
exhaustive search over all of the possibilities of results the remaining
matches can have.

\section*{Aims}

The primary aim of the project is to implement the aforementioned
Ford-Fulkerson algorithm that, given a league table, remaining schedule of
matches, and a team, will determine whether or not the given team has a
chance of finishing at the top of the league table. The results will then
be displayed on a user interface with data being obtained from either a text
file or parsing dynamic up-to-date information from a sports website. The user
interface will be fairly simplistic with a large central table displaying the 
league and elimination information with buttons and/or radio buttons for 
navigating between leagues. The user interface may also allow the user to 
navigate between weeks to see how the league progressed and when teams were 
first eliminated. A web-based version of the user interface is of interest and 
will be explored.

There may be various ways the given team can be eliminated from the league
and a secondary objective is to provide a `certificate' of elimination. This
will provide the team, or teams, that are responsible for ensuring the given
team cannot finish top of the league table. A simple example can be taken
from the na\"{\i}ve calculations traditionally used by sports pundits. The
teams responsible for the elimination is just the list of teams that have a
points differential with the given team that is greater than the number of 
games the given team has to play.

At time of writing, the algorithm has been completed and has been extended to 
show the teams responsible for eliminating a given team. The user interface is 
also largely complete as all currently available information is being displayed 
to the user. The previous/next week buttons currently offer no functionality as 
this is still to be implemented. The simple static text file parser is 
complete.

The next steps are to create a web-based interface that will pull data from a 
MySQL server's database on page load. This is currently at the functional 
prototype stage with a sample web page showing a simple textual output of all 
division league data but will soon be extended to four pages showing the data 
in a tabular format.

As an aside to the project, further statistical analysis will be discussed.
With baseball being one of the most statistically analysed sports it may be
possible to implement a certain confidence level indicator for teams which
still have a mathematically viable way of winning the league. If a given team
is known to win against the teams in the matches it has remaining then it can
be deemed more likely that the team can win the league, and vice versa. This
is not possible with the Ford-Fulkerson algorithm and is very much an interest
rather than a requirement.

\section*{Motivation}

The project is very interesting to me. I may have no knowledge or interest
in baseball itself however I do enjoy seeing algorithms and theories used in
a real world scenario.

The results that can be generated from the project are also interesting to 
baseball fans who may want to know how well their team is doing in the current 
league, and to see if any major competitors have been knocked out of the 
running. This will become increasingly relevant towards the end of the league 
as a clear winner, or possible winners, can become apparent.

\end{document}
