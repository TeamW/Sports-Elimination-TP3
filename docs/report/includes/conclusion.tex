\section{Summary}

\section{Future Work}

\begin{itemize}
\item Continue the implementation of the binary first non-trivial elimination
search to replace the working but inefficient linear search.
\item Implement a more efficient method for date emulation in the web
application.
\item Restrucure design of desktop based and web based user interfaces to take into account constructive UI design suggestions gained from the user evaluation process. 
 

\end{itemize}

\section{Lessons Learned}

Team communication 

At points durring the course of the project, there has been significant communication break downs. Although the team decide to use numerous ways to communicate (facebook, google docs, github tickets, team meetings and adviser meetings) smaller groupings of the team experienced significant breaks in communication durring important spells in the development cycle (christmas break), and unfortunately some messages between team members gots lost due to different team members focusing communcation efforts on different mediums.  To combat this, and using knowledge gained through experience,the team gradually reduced the amount of communication mediums it used as the project progessed. For future projects, the team would recommend keeping the amount of communication mediums used by the team to a minimum.      

Team structure 

At the start of the project the team decide on a laissez faire management structure, which it was thought would let individual team members gravitate towards areas of work in that suited their strengths. However, at the end of the first semster it was apparent that this was not a suitable mangement structure due to some team members having a far greater technical capability, personal knowledge \& effective work rate than other members of the team. It was decide that the team should restructured, \& one team meeting after the christmas break provided a suitable time for the team to vote on a Leader and Second in Command . It was decided by the team that Gordon Reid would be selected for the role of team leader, \& Ryan Wells for the role of . . . FILL IN  . For any future projects the team would recommend using a top down mamangement structure from the start of the project, to allow more technically experienced \& capable time managing memebers of the team skills to be drawn upon. And also to more easily highlight memebers of the team who may unfortunatlely take advantage of not having an explicit role by not contributing to the work flow to an acceptable level.           

Distrubution of work load

As mentioned above, a sizable concern for the durration of the project was the distribution of work load between team members. Durring the first few weeks of the development cycle it became clear that one or two members of the team had measurably greater technical capability and effective work rates than other team members. These team team members (One of whom who was chosen by the team as team leader, and the other as >>> FILL IN TITLE), were invaluable durring the course of the applications production life cycle, and again for the forward progression of the application as a finshed product. Although there must be a small skill gap between all team members in any team develpoment environment, usually this can be over came by time expendature, personal development and a good work ethic by team members who are less experienced. Unfortunately durring the development of the application, at numerous times, multiple team members had to be pulled away from there current role to assist one team member who it was felt universally was not pulliing their weight, or contributing to a satisfactory level. The team felt they delt with this in a suitably, durring numerous discussions with their adviser of studies.           

Measuring Scope 

An areas that the team felt like it could have preformed better on was measuring the scope of the project. Although great care was taken by the whole team durring the requirements gathering \& design stages of the system, the  team had the main aspects of the system in development completed and a working prototype to demonstrate to users before shortly after the christmas break. . . HELP WITH WORDING


Timing of evaluation 

The user evaluation itself was felt by the team to be a huge success, effectively communicating numerous positive aapects of the design of the system, and also shed light on aspects of the system which needed revising. Altough the team gathered numerous peices of constructive feedback from the user evaluation, the team unfortunately did not have time to consider implementing any positive design changes suggested by the user evluation due to the late point durring the project life cycle that the user evaluation took place. If the team was to do the project again, it would have started the user evaluation process a few weeks earlier, to leave room at the end of the project life cycle to potentially implement any constructive design suggestions gained from the user evaluation process.       

    


