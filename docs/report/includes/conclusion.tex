\section{Summary}

This project was very interesting and enjoyable. Taking knowledge of Java
and graphs and creating a system with a real-world application has been a
great experience. The non-trivial mathematics involved in the Ford-Fulkerson
algorithm was the most challenging part.

With this being the first group project for us all, the project acted as a
great experience for working with multiple people on a single project. We
faced some difficulties but were able to overcome them and produce what we
set out to do along with a couple of extensions and the web based application.

\section{Future Work}

\begin{itemize}
\item Continue the implementation of the binary first non-trivial elimination
search to replace the working but inefficient linear search.
\item Implement a more efficient method for date emulation in the web
application.
\item Restructure design of desktop based and web based user interfaces to
take into account constructive user interface design suggestions gained from
the user evaluation process.
 \end{itemize}

\section{Lessons Learned}

\subsection{Team communication}

At points during the course of the project, there has been significant
communication break downs. Although the team decided to use numerous ways to
communicate (Facebook, Google Docs, GitHub tickets, team meetings, and advisor
meetings) smaller groupings of the team experienced significant breaks in
communication. This was most apparent during important spells in the
development cycle, for instance the Christmas break. Unfortunately some
messages between team members were left unheard due to different team members
focusing communication efforts on different mediums. To combat this, and using
knowledge gained through experience, the team gradually reduced the amount of
communication mediums it used as the project progressed. For future projects,
the team would recommend keeping the amount of communication mediums used by  to
a minimum.

\subsection{Team structure}

At the start of the project the team decided on a Laissez-Faire management
structure, which it was thought would let individual team members gravitate
towards areas of work that suited their strengths. However, at the end of the
first semester it was apparent that this was not a suitable management
structure due to some team members having a far greater technical capability,
personal knowledge and effective work rate than other members of the team. It
was decided that the team should restructured. One team meeting after the
Christmas break provided a suitable time for the team to vote on a Leader and
Second in Command. It was decided by the team that Gordon Reid would be
selected for the role of team leader and Ryan Wells for the role of Second in
Command. For any future projects the team would recommend using a top down
management structure from the start of the project. This would allow more
technically experienced and capable time-managing team members' skills to be
drawn upon. This would also highlight members of the team who may  unfortunately
take advantage of not having an explicit leader by not  contributing to the work
flow to an acceptable level.

\subsection{Distribution of work load}

As mentioned above, a sizeable concern for the duration of the project was the
distribution of work load between team members. During the first few weeks of
the development cycle it became clear that one or two members of the team had
measurably greater technical capability and effective work rates than other
team members. These team team members (One of whom who was chosen by the team
as Team Leader, and the other as Second in Command), were invaluable during  the
course of the applications production life cycle, and again for the  forward
progression of the application as a finished product. Although there  must be a
small skill gap between all team members in any team development  environment,
usually this can be overcome by time expenditure, personal  development and a
good work ethic by team members who are less experienced.  Unfortunately during
the development of the application, at numerous times,  multiple team members
had to be pulled away from their current role to assist  one team member who it
was felt universally was not pulling their weight, or  contributing to a
satisfactory level. The team felt they dealt with this in a  suitably
professional and timely manner, usually during numerous discussions  with the
project supervisor.

\subsection{Measuring Scope}

Prior to starting the implementation of the project, multiple functional
requirements were stated using the MoSCoW approach (Must have, Should have,
Could have, Won't have (but Would like)). The team decided to `play it safe' and
have few `must haves' to keep the scope of the project within a definitely
manageable size. This led to the must have functionality being implemented and
demonstrable shortly after the Christmas break. For future projects it may be
beneficial to expand the scope of the project to challenge and spur on the
entire team to take on more work.

\subsection{Timing of evaluation}

The user evaluation itself was felt by the team to be a huge success,
effectively communicating numerous positive aspects of the design of the system,
and also shed light on aspects of the system which needed revising. Although the
team gathered numerous pieces of constructive feedback from the user evaluation,
the team unfortunately did not have time to consider implementing any design
changes suggested by the user evaluation due to the late point during the
project life cycle that the user evaluation took place. If the team was to do
the project again, it would have started the user evaluation process a few weeks
earlier, to leave room at the end of the project life cycle to potentially
implement any constructive design suggestions gained from the user evaluation
process.
