\subsection{Design}

In designing the desktop user interface the first step was to identify the main
goals of our users, in order to make sure that the system delivered the required
information quickly and clearly. After a discussion it was  agreed that the
typical user would have two main priorities - checking whether a specific team
had been eliminated and viewing the league tables for each division. This was
further broken down into a list of functional requirements.

\begin{itemize}
\item Display eliminated teams.
\item Display all divisions/leagues.
\item Display all teams in a division.
\item Automatically update the display on start-up.
\item Display certificate of elimination.
\end{itemize}

With these requirements in mind several paper prototypes were created by
different members of the team so as to increase the range of ideas we could
incorporate into the final design. These prototypes were compared in terms of
functional clarity and aesthetic appeal, with the effective elements of each
taken and incorporated into a final prototype design. This was the resulting
wire-frame.

\includegraphics[width=\linewidth,keepaspectratio]
{images/Prototype_UI.png}

\begin{enumerate}
\item League navigation panel - allow the users to quickly and easily select the
league and division. Deliberately laid out and grouped left to right to indicate
that divisions are subsets of leagues.
\item Standings table - immediately display the currently selected divisions major
statistics.
\item Loading bar - displays the progress of the Ford-Fulkerson algorithm.
\item Elimination indicator - indicate whether the team has been eliminated, both
trivially and non-trivially.
\end{enumerate}

This design focuses on serving the primary user needs by displaying the division
rankings and elimination status on start-up, which in the the best case scenario
would mean the user would not need to input anything to get the desired
information.

\subsection{Implementation}

The interface was written using the Java Swing framework - a popular tool-kit
used in building graphical user interfaces for Java programs. Initially the 
graphical Swing toolset Netbeans was considered, but quickly rejected after some
testing as it was simply too cumbersome for the tasks required.  The actual
building of the interface was a gradual and adaptive process, with the focus on
getting a working version of the  prototype up and running. This was
accomplished and, based on team feedback, features were added or removed. The
completed Ford-Fulkerson algorithm ran almost instantly, rendering the loading
bar unnecessary and so it was dropped. Date navigation buttons - including drop
down date selectors - were added, with a label displaying the currently selected
date. These  allow the user to move through the season anywhere between the
date/time of  the first and last matches of the season, respectively.

\includegraphics[width=0.9\linewidth,keepaspectratio]
{images/finalDesktopUI.png}

This final version of the UI features three main components - a table with
associated model,  two sets of radio buttons and the date navigation panel,
comprising of two buttons and three drop down boxes. The work flow of of the
radio and date navigation panels are similar - they each have event listeners that
set the appropriate data (the division/league and date respectively) 
according to the user input and call a shared function  (in order to minimise complexity). 
This update function gets the division chosen by the user and plays the 
matches up to the current/selected date. It then updates the table model, 
current date label and the date selection combo boxes, then redraws - now 
displaying the appropriate values, including an elimination status that is correct 
for that division and date. Another additional feature that was added after the base
prototype was completed is the ability to display the certificate of
elimination. Clicking on the `Eliminated?'' table cell will trigger a custom
mouse event listener, which opens a  text box displaying the elimination details
if they exist. This method of displaying the certificate was chosen as it
follows the principle of allowing the user access to as much key information
with as little input as possible, as it was decided that having to navigate
through a menu to find this  information would be unintuitive and unnecessary.
The ability to print to a pdf was also added later in the project and so the UI
was further extended to incorporate this. Since this functionality will be used
infrequently a more traditional file$-$$>$print menu was added, as the user will
almost certainly have used a similar command in other applications.
