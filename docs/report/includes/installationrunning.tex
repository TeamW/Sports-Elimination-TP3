\section{Desktop Application}

\subsection{Installation}

The desktop application requires the Java Runtime Environment (JRE) available 
from \url{http://www.java.com/en/download/index.jsp}. The application has been 
tested on JRE 6 and JRE 7 without issue.

The print functionality requires a LaTeX distribution that includes the
executable `pdflatex'. The installation procedure varies for each operating
system and instructions are available from
\url{http://latex-project.org/ftp.html}.

\subsection{Running}

After the Java Runtime Environment (JRE) is installed, running the application
only requires double clicking the supplied JAR file.

The print functionality is executed within the Java application and thus is
transparent to the user. In the event that the command `pdflatex' cannot be
found, the application will fail to print however will not crash. Print
functionality is known to work on standard installations of the distribution
on Linux/GNU-based and Mac OS operating system.

\section{Web Application}

\subsection{Installation}

Installation of the web application is not required as a remote host is running
the required software. This can be accessed via 
\url{http://www.gordonrenfrewshire.com/teamw}. For purposes of completeness
and satisfying the potential desires of the reader, an installation procedure
is supplied.

In the event that the supplied URL fails to work, please contact Gordon Reid
via any of the following methods:

Student email: 1002536r@student.gla.ac.uk

Personal email: gordon.reid1992@hotmail.co.uk

Mobile phone: 07706 477 672

The web server has numerous standard applications running to service the web
application. Each one is required for full functionality:

\begin{enumerate}
\item A web server (such as Lighttpd or Apache)
\item PHP (known to work on PHP 5.x)
\item Java Runtime Environment (version 6 or 7)
\item MySQL (version 5.x)
\end{enumerate}

\subsubsection{Installation of packages}

The installation procedure assumes you have super user access on a Debian-based
distribution. The official procedure for installation of a `LAMP' (Linux
Apache, MySQL, PHP) server is available from the Debian Wiki at
\url{http://wiki.debian.org/LaMp}

\subsubsection{Set up of database}

To create a database, login to the MySQL command line with `mysql -u root' and
execute the following commands.

\begin{verbatim}
create database teamw;
create user 'teamw'@'localhost' IDENTIFIED BY 'algorithms'
grant all privileges on teamw.* to teamw@localhost
\end{verbatim}

The user `teamw' can now log in to the database via the command:

\begin{verbatim}
mysql -u teamw -palgorithms teamw
\end{verbatim}

Afterwards, to create the required tables, the below SQL query needs to be
executed for the six divisions (in place of the $<$division name$>$).

\begin{verbatim}
CREATE TABLE `<division name>` (
	`Team` varchar(25) NOT NULL,
	`Points` int(4) NOT NULL DEFAULT 0,
	`Games Played` int(4) NOT NULL DEFAULT 0,
	`Eliminated' tinyint(4) NOT NULL DEFAULT 0,
	PRIMARY KEY ('Team')
)
\end{verbatim}

In the folder `website/content/php/includes/functions.php' there are a number
of variables at the top of the page indicating the values for the server, user,
password, and database. These can be modified to suit your requirements however
the default are highly recommended.

A word of warning, the variable scope is very insecure and however was designed
as such for simplicity of installation and testing. Please do not run the server
code on a public or production server.

\subsection{Running}

As stated in the installation section, the web application is available for
viewing at \url{http://www.gordonrenfrewshire.com/teamw}. If a personal
installation has been executed then running the application will be dependent
on your own set up.

If a personal setup has been created, you will need to run the `Update
Divisions' page found at the navigation bar to populate the tables.