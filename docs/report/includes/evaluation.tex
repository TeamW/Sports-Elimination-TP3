\section{Evaluation}

\subsection{Introduction}

After completing the implementation of the desktop and web based applications a thorough user evaluation was carried out of both systems. The user evaluation functioned as a reliable way to test the usability and likeability of both system designs. All usability testing preformed by the team was done in accordance with the University of Glasgow ethics procedures. The Evaluation was preformed in a three stage process. 


\subsubsection{ Participant Brief }

Starting with an introductory briefing, the team member conducting the evaluation introduced the test participant to the system, provided the user with a test number and a copy of the user test script and described the aims of the user evaluation that was about to take place.  

During this stage the participant was asked to answer a few simple questions to gauge their competency using desktop and web based applications, and to gauge their personal interest in the systems domain -sports statistics- .

 During the introductory brief, it was made clear to the participant that no personal or identifying information would be collected from them during the user evaluation.  It was decided by the team that this would hopefully decrease the amount of participants who would not complete the evaluation fully, and to ease the process of gaining eithical permissions from the university . Due to the fact that no test participants decided to stop the evalation half way through, and gaining ethical approval for the user evaluation was a simple process, it was felt like this was a benifical desion . 

In accordance with the universities ethical prodedures, during the introductory breifing the test participant were reminded of their right to stop the evalution at any time with no requirement to give reason. The user was then further reminded that it was not them, but the system that was under evaluation, and the user was provided with the contact details of the team member conducting the evaluation, to allow the user to contact the team to answer any question or after thoughts that they had about the system, or the user evaluation after it had been completed and given some time to think on the process .

\subsubsection{Think-Aloud (usability) }

The evaluation was performed using the ”Think-Aloud” technique. The test participants were encouraged to talk out loud as they preformed a series of tasks, designed to provide a full overview of the complete functionality provided by both systems. 

At this stage the role of the team member conducting the evaluation was to observe the test candidates interaction with the system, and take note of any hesitation, possible confusion, or errors encountered when using the system. The reactions shown by the test participant when interacting with both the web based and desktop based applications clearly highlighted usability problems which have went unseen in the teams initial system design. 


\subsubsection{Questionnaire (likeability)}

The final part of the user evaluation asked the test canddidate to complete a feedback questionare. This document asked them to rate their intest in the complete system after thier inital experience, (FINSH THIS SENTENCE )    . At this stage the test candidates were also given the opportunity to ask any further questions about the each of the systems. After being thanked for their time and made aware of the tests completion, every participant was ecouraged to get in contact with a member of the team if they had any further thoughts the wished to add on the system after having some time to think about the evaluation process.    

\subsubsection{4.1 Desktop Application}

\begin{table}[t]
\centering
\begin{tabular}{|l|c|c|c|c|c|}
\hline
Product & 1 & 2 & 3 & 4 & 5\\
\hline
Price & 124.- & 136.- & 85.- & 156.- & 23.-\\
Guarantee [years] & 1 & 2 & - & 3 & 1\\
Rating & 89\% & 84\% & 51\% & & 45\%\\
\hline
\hline
Recommended & yes & yes & no & no & no\\
\hline
\end{tabular}
\caption{This is a table template}
\label{tab:template}
\end{table}


TURN THIS INTO A TABLE !

1.Increase visibility of league and division selection

2.increase visibiity of certifiacte of elimination

3. change window open name print doc header

4. provide user manul for print functionaliy

7 provide hover over more info details for print functionality options

5. provide "new step" pop up after new league genaration to indicate league has to be LOADED into system .

6. provide user manual for generating lagu


\subsubsection{Evaluation Results}

\subsubsection{4.2 Web Application}

\subsubsection{Evaluation Results}

















































