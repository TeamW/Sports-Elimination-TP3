\subsection{Design}

\subsubsection{Introduction}

This section discusses the design of the web-based version of the application.
The design of the application was constructed with only the most important
functional and non-functional requirements in mind. The web-based version was
classed as the least important part of the project, with preliminary effort
directed solely towards the desktop application.

\subsubsection{System architecture}

The web application is a standard multi-tier architecture with the presentation,
logic, and data separated from each other.

The presentation tier is the client/browser who has Hyper Text Mark-up Language
(HTML) and Cascading Style Sheets (CSS) for the static presentation of content.
In addition there is JavaScript supported by JQuery and JQueryUI for the dynamic
user interface elements.

The logic tier runs on a web server called Lighttpd (pronounced lighty) that is
supported by PHP: Hypertext Preprocessor (PHP). The logic tier has two data
sources that make up the data tier, a MySQL database containing the latest data
and a Java jar for looking back at older data.

The N-Tier Architecture diagram is available from appendix
\ref{fig:nTierArchitecture}.

\subsubsection{User interface}

The user interface of the web application was intended on being as close to the
desktop interface as viable within the constraints of a web browser and within
the realms of what is a typical layout of a web page.

A wireframe for the web application is shown in appendix
\ref{fig:webApplicationWireframe}.

The web application has a single page containing the six available divisions.
Each division is a table and only one is available for viewing at a time. The 
reasoning behind this is to keep as much information `above the fold' (above the
lower page boundary on a browser's window).

There are links at the top of each page that will allow the user to traverse
the entire date range for the season allowing them to view the scoreboard and
elimination status at any point in time.

\subsection{Implementation}

\subsubsection{Introduction}

This section discusses the implementation of the web-based version of the
application. The implementation discussion will be split up into the three
main sections as shown in appendix~\ref{fig:nTierArchitecture}.

\subsubsection{Presentation - Client/Browser}

The website uses three of the most common web technologies in use: HTML, CSS,
and JavaScript (supported by jQuery and jQuery UI).

\subsubsection{Logic - Web Server/PHP Processor}

The website is dynamically created by using PHP. PHP interfaces with data one
of the data layer sections and, from the results obtained from the data layer,
produces the division tables and pushes the generated HTML along with the
attached CSS and JavaScript.

\subsubsection{Data}

The web application has two primary sources of data: a MySQL database and a
Java JAR file. The MySQL database contains the recent information available
on the sports divisions whereas the Java JAR contains a text file of the entire
result set allowing the user to request to see the state of a division at any
point in the league.

\subsubsection{Data - MySQL Database}

The database is structured with six tables, one for each division. Each table
structure is identical with columns for the team name, the number of points
a team has, the number of games they have played, and whether or not they have
been eliminated from the division. This allows for a direct mapping from
database storage to the user's view. The structure also greatly simplified 
database queries as information from an entire division was a single simple, and 
thus efficient, SQL query. An example query to see the league table, ordered by 
number of points for the American West division is shown below:

\begin{verbatim}
SELECT *
FROM `American West`
ORDER BY Points DESC;
\end{verbatim}

This database is updated with the latest division statistics supplied by the 
Java JAR as the JAR contains the algorithm to compute the elimination status.
The update can be completed in two ways. If the division is new and thus no
information is present in the table an INSERT INTO statement is performed for
each time. If the team is present within the database (as found via a SELECT
statement), an UPDATE is completed instead. Examples of the three queries
for the American West division team `Los Angeles Angels' are shown below:

\begin{verbatim}
SELECT *
FROM `American West`
WHERE Team = `Los Angeles Angels';
\end{verbatim}

\begin{verbatim}
INSERT INTO `American West`
       (Team, Points, `Games Played`, Eliminated)
VALUES (`Los Angeles Angels`, 89, 162, 1);
\end{verbatim}

\begin{verbatim}
UPDATE `American West`
SET Points = 89,
    `Games Played` = 162,
    Eliminated = 1
WHERE Team = `Los Angeles Angels`;
\end{verbatim}

\subsubsection{Data - Java JAR}

The JAR file has embedded within it a text file containing the entire list of
results from every match in the baseball season being discussed. This is parsed
on-demand up to the date specified by the user. This allows users to ask to see 
the results and associated elimination status at any point in time. This is not 
the most efficient method of implementation however the web application as a 
whole was an extension to the primary aims of the project and as a result wasn't 
started until the second semester after the desktop application had been 
significantly implemented. With the time constraints and significant complexity 
of a more efficient implementation it was decided that this was the best 
solution with the current resources. A more efficient implementation was 
assigned to the future work section of the project.
