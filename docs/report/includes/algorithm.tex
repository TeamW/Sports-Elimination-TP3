\subsection{Terminology}
Residual Graph

Residual Path

Breadth First Search

Backwards/Forwards Edges

Ford-Fulkerson Algorithm

Flow/Capacity

Maximum Flow

Node/Vertex

Max Flow-Min Cut

Certificate Elimination

\subsection{Design}

Wayne paper~\cite{Wayne}

Kern paper~\cite{Kern}

Adler paper~\cite{Adler}

The design of the algorithm is centered around a directed graph of
team nodes and match nodes as is described inside the Wayne
paper~\cite{Wayne} and tests the elimination status of $team_k$. The
graph is created with vertices being both the team and match nodes - a
team node symbolises one team in the division and a match node
symbolises the matches between $team_i$ and $team_j$. Each match node
has an outward edge to both $team_i$ and $team_j$ of an infinite
capacity and an inward edge from the source of capacity equal to the
number of games remaining between $team_i$ and $team_j$. Each team
node has an inward edge from their respective match nodes and an
outward edge to the sink of capacity equal to the maximum number of
points that $team_k$ can achieve if they win all their remaining games
minus the number of wins the team belonging to the respective team
node has. 

It was also required that the algorithm should host the functionality
to create a residual graph of the graph it is currently working on and
then find a residual path on this residual graph.

The methodology used to evaluate $team_k$'s elimination status is as
represented in the Wayne paper~\cite{Wayne} and implemented as the
pseudocode representation below. 

It was decided to store the elimination status of a team inside the
respective Team object so the algorithm will only be run if the team's
elimination status is ``not eliminated''(boolean false). This is a
decision for pure optimisation and is not part of the Ford-Fulkerson
algorithm.

Two ``short circuts'' have been added to the algorithm for efficiency:
if $team_k$ has no remaining games and is not at the top of the league
then $team_k$ is trivially eliminated; if the difference between
$team_k$'s current points and the league leaders current points is
more than the matches $team_k$ has to play then the $team_k$ is
trivially eliminated.

It was decided that the algorithm should have a verbose flag to output
its computation every iteration - this was to help debugging and also
allow the reader to analyse the algorithm at various stages of its
cycle. 

\subsection{Implementation}

- Pseudocode



\begin{algorithm}[H]
                               
  \SetAlgoLined
  
  \KwData{this text}
  
  \KwResult{how to write algorithm with \LaTeX2e }
  
  Create network graph.
  
  \ForEach{Vertex in graph}{
    \ForEach{AdjListNode in Vertex}{
      Set flow of AdjListNode to zero
    }
  }
  Capacity of source 
  
  \caption{THIS IS A CAPTIONN}
\end{algorithm}