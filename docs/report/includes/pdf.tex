\subsection{Design}

To enhance the user experience it was decided that a user should be
able to save certain moments of interest inside a league, for example
elimination events, end of league results or any arbitrary day of
interest to the user. The major aims of this module were to try and
cover the needs of as many users as possible, have an elegantly
formatted output, and to have an easily accessible process that
obscured the inner workings of the printing process as to not distract
the user from the main functionality of the application.

We discussed several options that would solve our problem:

\begin{itemize}
  \item[Physical Printing] Direct physical printing of the
    output. Assumes that the user wants a physical printout of the
    chosen screen and does not cater for users whom this is not their
    intentions or those who would wish to share electronically. It is
    outside the scope of the project to create fully integrated
    physical printing so we instead considered creating a PostScript
    document and then prompting the user to print the document who
    would then have to manually deal with finishing the printing.
  \item[Text File] Easily achievable to create an ASCII interpretation
    of what the user would like to save. Not as nicely formatted as
    some of the other options and overall not as impressive as we
    would like this module to be.
  \item[Emailed results] Similar to the Text File in terms of
    formatting but with the added disadvantage of forcing a user to
    be on-line to receive the results and creating an email dispatcher
    may escalate beyond the resources allocated for this project.
  \item[\LaTeX] Export a PDF of the information the user wishes to
    save. Nicely formatted and the user can then do what they wish
    with the PDF. Caters to the needs of all users, however the user
    must have an installed version of \LaTeX to utilize this as
    packaging a \LaTeX installation within the module would be far too
    bloating to the overall application.
\end{itemize}

After discussion of the problems we may counter with each option and
analysing the implications in a multi-platform manner, it was decided
that integrating with \LaTeX was the most viable option to best suit
our needs. The module has been designed to work with a standard
installation of \LaTeX with no modifications necessary. It was agreed
that for the scope of the project it was a reasonable requirement of
the user that a standard installation of \LaTeX be present on the host
machine.

The module has been designed in such a way that there is no limit on
the quantity of elements the user can print (except for the memory
implications of the Java Virtual Machine) and the module is fully
extensible if other printable features were to be added to the
application.

Each printable element is stored in an ordered manner and
the entire document is saved upon the users request. It was agreed
that a user should also be able to print one screen instantly without
having to add this element to the ordered store and then request it to
be printed. This was a decision with the implications of improving the
usability and user experience of the module and combines both the
standard steps of adding to the ordered store and requesting the print
in one fell swoop.

As a user navigates through the league they can add any table to the
ordered store from any division at any time. There is only one ordered
store, and hence only one printable document active at any one time,
this is to save confusion and make the entire process easier to use.

\subsection{Implementation}

The ordered store is implemented as a LinkedList of
DocumentSections. A DocumentSection is an abstract class which will
be implemented by any element of the application that wishes to be
printed. The notable attribute of the DocumentSection is a StringBulder
which holds the physical lines of \LaTeX code and content to provide
the fully formatted content. This is a protected element and it is
intended that any subclass of DocumentSection is directly responsible
for formatting the contents of the StringBuilder in such a way as to
not cause compile errors that require user interaction - these are
uncaught by the application and may cause unwanted behaviour. The
decision to make the StringBuilder protected is also to remove the
risk of another entity editing the contents in such a way as to
introduce compile errors and as such any subclass of DocumentSection
should not allow direct manipulation or access to its StringBuilder.

The implemented subclasses of DocumentSection that are available to
the application allow the user of the application to add as many
division tables, on any date and precisely as they appear on the
screen, to the ordered store as they wish. Row sorting is maintained
and the implementation of this functionality is fully self contained
requiring only the Table model hosting the data and the Date that
relates to the data in the Table.

When the user has decided they have added all the information they
wish to export they can invoke functionality to export the document
they have set up. This requires only a file name that the exported
document will be named after and the directory to which this shall
reside in. It is worth noting that the user who invokes the
application must have access to the directory and have file creation
permissions on the directory they wish to write in.

A StringBuilder is used again to concatenate the individual
DocumentSections together into one String and a standard \LaTeX
document start and end are appended to the beginning and end of the
StringBuilder respectively. This is then written to a TeX file inside
the application directory to be compiled in the final stage.

The final stage of this process is to compile the \LaTeX document to
PDF, remove other files generated by \LaTeX and then move the final
PDF to the directory of the users choice. Each of these steps require
a Runtime Process to execute as the application needs access to the
command line as a suitable Java library could not be found. The full
completion of each process happens before the next process will
continue and before control is handed back to the user - it is for
this reason that any implementation of DocumentSection must be error
free as the application will become unresponsive while the Process
running \LaTeX waits for input from the user that it will never
receive. Interaction could be added to either ask for input from the
user or to cancel the document entirely but a decision against this
was made as to increase the user experience of this functionality.

It is intended that the user should be able to export other elements
of the document in the future - the core implementation of the export
module lends towards extensibility and so other sections of interest
have the potential to become exportable in due course.
