\section{Motivation}

The Algorithms for Sports Elimination project assigned to us, Team W, was a
great opportunity to see the theory of algorithms and graphs used in a real
world context.

The results that can be generated from the project as also interesting to
baseball fans who may want to know how well their team is doing in the current
league, and to see if any major competitors have been knocked out of the
running. This will become increasingly relevant towards the end of the league
as a clear winner, or possible winners, can become apparent.

\section{Background}

The project builds on two papers: one by K. D. Wayne, `A new property and a
faster algorithm for baseball elimination', SIAM Journal on Discrete
Mathematics \url{http://epubs.siam.org/doi/abs/10.1137/S0895480198348847}
\cite{Wayne} and one by I. Adler, A. L. Erera, D.S. Hochbaum and E. V. Olinick,
`Baseball, optimization and the world web', manuscript, 1998
\url{http://riot.ieor.berkeley.edu/~dorit/pub/baseball.ps} \cite{Adler}.

The project came about from the American love for baseball, and the love for
a multitude of statistics relating to baseball. A common news story for sports
fans is the announcement that a team can no longer win the current season,
however the calculations used to work this out are typically na\"{\i}ve and
do not take into account the remaining schedule of games. Instead it is
usually just the difference between the points the top team has and the number
of points plus remaining games another team has. If the number of points
for the top team is the bigger number, the other team has been eliminated.

The papers stated above build on the na\"{\i}ve calculations and make use of a
simple, yet effective, algorithm to reliably answer the question:
`Is it mathematically impossible for a given team to win the baseball league?'
The algorithm is known as the Ford-Fulkerson algorithm, named after L. R. Ford,
Jr. and D. R. Fulkerson. The Ford-Fulkerson algorithm removes the need for an
exhaustive search over all of the possibilities of results the remaining
matches can have.

\section{Aims}

The primary aim of the project is to implement the aforementioned
Ford-Fulkerson algorithm that, given a league table, remaining schedule of
matches, and a team, will determine whether or not the given team has a
chance of finishing at the top of the league table. The results will then
be displayed on a user interface with data being obtained from either a text
file or parsing dynamic up-to-date information from a sports website. The user
interface will be fairly simplistic with a large central table displaying the
league and elimination information with buttons and/or radio buttons for
navigating between leagues. The user interface may also allow the user to
navigate between weeks to see how the league progressed and when teams were
first eliminated. A web-based version of the user interface is of interest and
will be explored.

There may be various ways the given team can be eliminated from the league
and a secondary objective is to provide a `certificate' of elimination. This
will provide the team, or teams, that are responsible for ensuring the given
team cannot finish top of the league table. A simple example can be taken
from the na\"{\i}ve calculations traditionally used by sports pundits. The
teams responsible for the elimination is just the list of teams that have a
points differential with the given team that is greater than the number of
games the given team has to play.

\section{Outline}

The remainder of the report will cover many aspects of the project including:

\begin{itemize}
\item Design and Implementation - The project is split into several different
sections. The design and implementation of each section will be discussed in
turn. For the Ford-Fulkerson algorithm section there will also be a greater
discussion of how graph theory and network flow relate to the sports
elimination calculation.
\item Evaluation - The project's two primarily methods of evaluation will be
discussed: the correctness testing completed by the team, and the user
evaluation.
\item Conclusion - For the last main section of the report the overall
project and future work will be discussed.
\end{itemize}

At the end of the report the individual contributions and team structure will
also be stated for the purposes of completeness.