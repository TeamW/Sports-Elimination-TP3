\section{Terminology}

The web application design and implementation section discusses many
web technologies along with their acronyms. The acronyms will be
described with a brief overview of the technology.

\begin{itemize}
\item HTML: Hyper Text Markup Language. HTML is the basis of every web page
structure. It is a structured document with tags denoted by angle brackets (such
as $<$html$>$). These tags form a nested, tree-like structure that is read by the
browser to create the page layout of the website.
\item CSS: Cascading Style Sheets. CSS provides the style information for the
HTML document. CSS enables the separation of the HTML document content from the
document's presentation and defines a consistent approach for specifying the style
of an HTML document.
\item JS: JavaScript. JS is an interpreted computer programming language used by
browsers to provide dynamic, interactive web page content without requiring
communication with the web server. JS is bundled alongside an HTML document.
\item jQuery and jQuery UI: These are libraries (bundles) of JavaScript code to
simplify the development of JavaScript.
\item PHP: PHP Hypertext Preprocessor. PHP is a server-side computer programming
language used to generate dynamic web pages prior to sending the web page on to the
browser in the form of HTML. PHP may interact with databases and other external
data sources during the creation of the web page.
\item SQL: Structured Query Language. A high-level language for expressing what
a user wants from a database.
\item MySQL. The relational database management system that manages some data set
represented by tables. MySQL also accepts SQL queries and will return data that
matches what the SQL query requests.
\end{itemize}

This report discusses graph theory and network flow relating to the
Ford-Fulkerson algorithm in significant depth. This is discussed in a separate
section, section \ref{sec:graphTheoryAndNetworkFlow}.

\section{Graph Theory and Network Flow}
\label{sec:graphTheoryAndNetworkFlow}

Graphs are made up of two components: vertices and edges. Vertices can be
thought of as cities, with edges being roads between cities. A road between
two cities can have a maximum number of cars on it at one time. This is
known as the edges capacity. The road has a number of cars on it at present
time, known as the edges flow. The flow of traffic can never exceed the
capacity of the road.

The road from city A to city B is separate from city B to city A and the
existence of one road does not imply the existence of the other.

Network Flow is the study of working out how to have as much traffic as possible
move from a start city (the source) to an end city (the sink) by passing through 
intermediate cities. By using this concept, it is possible to determine
the elimination of teams in a sports league. This application of graph
theory and network flow will be discussed in detail in section
\ref{sec:fordFulkersonAlgorithm} on the Ford-Fulkerson algorithm.
