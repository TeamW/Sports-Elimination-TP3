\section{Terminology}

This report discusses graph theory and network flow relating to the
Ford-Fulkerson algorithm in significant depth.

\section{Graph Theory and Network Flow}

Graphs are made up of two components: vertices and edges. Vertices can be
thought of as cities, with edges being roads between cities. A road between
two cities can have a maximum number of cars on it at one time. This is
known as the edges capacity. The road has a number of cars on it at present
time, known as the edges flow. The flow of traffic can never exceed the
capacity of the road.

The road from city A to city B is separate from city B to city A and the
existence of one road does not imply the existence of the other.

Network Flow is the study of working out how to have as much traffic move
from a start city (the source) to an end city (the sink) by passing through 
intermediate cities. By using this concept, it is possible to determine
the elimination of teams in a sports league. This application of graph
theory and network flow will be discussed in detail in section
\ref{sec:fordFulkersonAlgorithm} on the Ford-Fulkerson algorithm.
