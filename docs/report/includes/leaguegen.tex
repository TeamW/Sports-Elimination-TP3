\subsection{Design}
The necessity of having the ability to generate new leagues arose
after the algorithm was created. Prior to this point our oracle for
testing the algorithm was the information found inside the Wayne paper
\cite{Wayne} and the standard data the application will load with
(2011-2012 league information), this was agreed to be not enough
rigour to fully test the algorithm and so the league generation was
created. After further discussion it was agreed that this
functionality was to be added to the final desktop application as it
would: increase the user experience of the application, enhance the
testable functionality of the application and provide the potential
for a user who is interested in sports elimination calculations to
manipulate the generated file to further examine or create elimination
criteria.

The major design decision of the generate functionality was to
generate a league in such a way that every team in the league,
regardless of division, will always play the same number of games and
therefore have an equal chance of finishing top of a given
division. This would be implied in a league where every division has
an equal number of teams but the scenario we have based our
application around has divisions with between four and six teams per
division. A team should not have a bias depending on the league it
plays in, and with our best efforts this should not occur. Other
modules of the application depend on the league being in a very
specific format otherwise a completely new league with an equal number
of teams per league could be generated without having to care for
these preconditions. 

The file created by the generate functionality must be in the same
format as the original league given to the application on startup due
to the parser and utilising existing functionality inside the
application this was easily achieved.


\subsection{Implementation}
