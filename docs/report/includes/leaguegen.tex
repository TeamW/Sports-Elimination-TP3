\subsection{Design}
The necessity of having the ability to generate new leagues arose
after the algorithm was created. Prior to this point our oracle for
testing the algorithm was the information found inside the Wayne paper
\cite{Wayne} and the standard data the application will load with
(2011-2012 league information). The team agreed that this would not be
enough to fully test the algorithm thus the league generation was
created. After further discussion it was agreed that this
functionality was to be added to the final desktop application for the
reasons as listed below:

\begin{itemize}
\item To increase the user experience of the application.
\item To enhance the testable functionality by supplying different data
sets.
\item To allow users interested in sports elimination calculations to
manipulate the generated file to further examine or create
elimination criteria.
\end{itemize}

The major design decision of the league generation functionality was to
generate a league in such a way that every team in the league,
regardless of division, will always play the same number of games. This
gives all teams an equal chance of finishing top their respective
divisions. Equality would be implied in a league where every division has
an equal number of teams but the real-world scenario is the six divisions
have a varying number of teams with one division having four teams, another
having six, and the rest having five.

It was decided that, since each division has a variant number of teams, a
team in a given division should not have the chance to perform better than
another team in a different division if the number of teams in each of
these divisions are not equal - that is to say your odds of finishing
top of a league are not better or worse than another team in a
different league that has a greater or lesser number of teams than
yours. With our best efforts this should not occur in the output of
the generate functionality. Other modules of the application depend on
the league being in a very specific format otherwise a completely new
league with an equal number of teams per division could be generated
without having to care for this decision.

One necessity of the generate functionality was to generate a league
with a time scale that was roughly the same season length as a real
season. This would create a more realistic analysis for any user that
has already perused the standard data provided when the application
first starts.

The file created by the generate functionality must be in the same
format as the original league given to the application on start-up due
to the parser requiring a specific format and to utilise existing
functionality inside the application.

\subsection{Implementation}
The total number of games picked for each team to play was computed to
be 180 games per team. This was not an arbitrary choice as each team
in the entire season must play the same number of games and the season
should be of a similar length to an actual season.

For each team to play the same number of games, the Lowest Common
Multiple (LCM) of the variant sizes of leagues must be found. The
variant sizes are 4, 5 and 6 and the LCM therefore is 60. This then
implies that any integer multiple greater than or equal to 1 of the
LCM will ensure that every team will play the same number of games
regardless of league. The actual number of games played by each team
in the league is 162 and the closest multiple of the LCM to 162 is 180.
The closest multiple is chosen with the intention of this resulting
in a season length similar to an actual season length. The reason for the number
of games being 162 is complicated (and regularly changes), it is left to the
reader's desire if they wish to read about the baseball scheduling. A brief
discussion and history is available from

\url{http://en.wikipedia.org/wiki/Major_League_Baseball_schedule}

Both the first and the second iteration of the generation
functionality follow the same game creation and scheduling
pattern.~\ref{GENLEAG}


\IncMargin{2em}
\begin{algorithm}
  \SetAlgoLined
  \SetKwData{Division}{Division} \SetKwData{fix}{``fix''}
  \SetKwData{Team}{Team} \SetKwData{TeamI}{TeamI}
  \SetKwData{TeamJ}{TeamJ} \SetKwData{Fixtures}{Fixtures}
  \SetKwData{League}{League} \SetKwData{Match}{Match}
  \fix $\leftarrow$ Store of (Team,Score increase) pairs\;
  \Fixtures $\leftarrow$ Store of created matches\;
  \Begin{
      \ForEach{\Division in \League}{
        Establish the \fix values for each \Team in \Division\;
        \ForEach{\TeamI, \TeamJ in \Division}{
          \ForEach{\Match between \TeamI, \TeamJ}{
            Set score for \TeamI and \TeamJ w.r.t \fix value for each team\;
            Format a \Match line to suit parser\;
            Insert line into \Fixtures at a random location\;
          }
        }
      }
    }
    Semi-periodically insert a change of day into \Fixtures\;
    \BlankLine
    \caption{League Generation Algorithm}\label{GENLEAG}
  \end{algorithm} \DecMargin{2em}


A StringBuilder was used to hold the generated lines that will become
the new file - this was a decision for performance reasons due to the
line count being large enough that standard string concatenation would
be inefficient.

It was quickly noticed in the first iteration of the generate
functionality that the occurrences of non-trivially eliminated teams
were scarce in the entire league and it was often the case that no
eliminated teams were found. The distribution of scores for every team
in a division lies within a small margin (assuming that each team has
played an equal number of games at this time) due to the way that
scores are assigned. Each team in a match is assigned a different
score that is randomly generated independent of the other teams
score. The implications of assigning a large number of games in this
manner is that for any two teams in a division the distribution of
victories will tend towards an even split for a large enough number of
games - this is true of the first iteration of the generate
functionality. A non-trivial elimination tends only to happen later in
the season and by this time the scores of each team in a division lie
within a small distribution. The implications of this are that
occurrences of non-trivally eliminated teams are low due to the manner
in which the Ford-Fulkerson algorithm works.


The strive to create a fair league had inadvertently created one in
which we could not present the interesting results discovered by our
application and so it was decided to ``fix'' the results generated so as
to demonstrate and further test the full capacity of the application.

% Ryan:
% The hashmap is not to have a unique identifier for each team. It
% just so happens that the way the fixing is done creates a unique
% value for each team but ONLY in the division it plays in, not in the
% overall league. This does not need to be the case, it is merely the
% choice of numbers I picked. Changed back, check the code if unsure
% of what I mean.
The method of ``fixing'' the results was to give certain teams in a
league a constant score increase for each game played.
Prior to the fixtures of a league being created, a HashMap of each
Team object in the league to an Integer value is created - it is
these values that signify the score increase for each team. As each
Team is added into the HashMap the integer value associated with that
Team is one more than the current maximum of all Teams in the selected
Teams division, with a starting value of zero for the first Team in
each division. When the matches are being generated the integer value
associated with each Team in the match is added onto the respective
Teams score - for example if $team_k$ was assigned a score of 6 and the
Integer associated with $team_k$ in the HashMap was 4 then $team_k$'s
score for this match will now be 10. This manipulation of the results
has the potential to grant victory to a team that would normally have
lost the match.

This ``fixing'' of results now generates a league that frequently has
non-trivial eliminations far more often than none at all and it was
decided that this was a better choice for the generate functionality
as it improved the overall user experience of this segment of the
application. It is worth noting that another method of score increase
may more accurately model an actual leagues elimination
characteristics but this one was chosen to explicitly demonstrate the
algorithms capabilities.
