\section{Correctness Testing}

Due to the complexity of the Ford-Fulkerson algorithm and the lack of a
publicly available test oracle, the majority of the correctness testing had to
be done manually. To further complicate matters each division has around four
hundred matches, making correctness testing a laborious task.

Some automated testing using the automated Java test suit, JUnit, was possible.
For the parser the only available test was ensuring that the parser read in
every single match supplied and that, in the case of our default test file,
all matches were registered as being played. The parser passed this test.

There was a small example test data from the Wayne paper~\cite{Wayne}. That
contained a division part way through the season that contained four teams.
Two of the teams weren't eliminated with another trivially eliminated, and the
other non-trivially eliminated. With the results of the elimination calculation
known to us, a small number of tests could be created to test the algorithm.
The algorithm passed each of these tests.

The manual correctness testing of the algorithm involved drawing out the graphs
the Ford-Fulkerson would be expected to create, along with the appropriate
edge capacities. By stepping through the explanation of how the Ford-Fulkerson
algorithm worked in the Wayne paper~\cite{Wayne} the residual graphs were
created and the values of the residual capacity were added to the appropriate
edges on the original drawn graph. Once this could no longer be done, manual
inspection of the flow was done to determine whether or not this is saturating.
The result was then matched up with the displayed data from the desktop
application. All manual calculations matched up with the algorithm output.